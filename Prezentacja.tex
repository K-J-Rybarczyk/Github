\documentclass[9pt]{beamer}


\mode<presentation> {
	
	\usetheme[alternativetitlepage=true
		      ]{Torino}

}

\usepackage[]{polski}
\usepackage[utf8]{inputenc} 
\usepackage{graphicx}
\usepackage{booktabs}

%----------------------------------------------------------------------------------------
%	TITLE PAGE
%----------------------------------------------------------------------------------------

\title[To w tym szablonie i tak nie działa]{\Huge Github, czyli tam i z powrotem z repozytorium}

\author{Radosław Głombiowski\\oraz\\Karolina Rybarczyk} 
\institute{
Uniwesytet Gdański\\ 
\medskip
\textit{predoobcy@gmail.com, szprota@rybki.com.pl} 
}
\date{} % To tutaj to po to, by usunąć tą cholerną datę, która z automatu się pojawia.

\begin{document}

	\begin{frame}
		\titlepage
	\end{frame}



	%----------------------------------------------------------------------------------------
	%	PRESENTATION SLIDES
	%----------------------------------------------------------------------------------------



	\begin{frame}
		\frametitle{Czym jest Github?}
		Seriws Github powstał w kwietniu 2008 roku. Jest portalem umożliwiającym hosting projektów informatycznych. Udostępnia tworzenie darmowych, publicznych repozytoriów (ma też odpłatną funkcję repozytoriów prywatnych), ułatwia pracę zespołową, prezentuje statystyki danego repozytorium oraz wiele więcej.\\
		W roku 2009 Github ogłosił, iż na stronie istnieje 90 000 unikalnych repozytoriów. Najnowsze dane (23 grudnia 2013) mówią już o istnieniu 10 000 000 repozytoriów.
		
		
		
		
		\begin{figure}
			\centering
			\includegraphics[height=3cm]{Octocat.png}
		\end{figure}
		
		
		
	\end{frame}
	

	%------------------------------------------------

	\begin{frame}
		\frametitle{Raczkowanie z Githubem}
		\begin{itemize}
			\item Założenie konta na portalu Github.
			\item Przypisanie klucza ssh do naszego konta Github.
			\item Utworzenie repozytorium na portalu oraz przyłączenie go do naszego lokalnego repozytorium.
		\end{itemize}
	\end{frame}

	%------------------------------------------------

	\begin{frame}
		\frametitle{Spis podstawowych komend do obsługi repozytorium}
			\begin{block}{git init}
				Zainicializowanie repozytorium git w folderze, w którym się znajdujemy.
			\end{block}

			\begin{block}{git add .}
				Dodaje wszystkie zmienione pliki do commita.
			\end{block}
			
			\begin{block}{git rm "nazwa pliku"}
				Usuwa podany plik z repozytorium/commita. \\UWAGA! Nie usuwa pliku z folderu a jedynie z listy plików, które będą przesłane na zdalne repozytorium.
			\end{block}			


	\end{frame}

	%------------------------------------------------
	
		
	\begin{frame}
		\frametitle{Spis podstawowych komend do obsługi repozytorium}

			\begin{block}{git commit -m "Treść"}
				Utworzenie commita. Rzecz niezbędna, w treści warto ogólnikowo podać, jakie zaszły zmiany w naszym repozytorium.
			\end{block}

			\begin{block}{git remote add origin git@github.com:"Nasz login Github"/"Nasze repo".git}
				Ustalenie konkretnego adresu dla naszego zdalnego repozytorium.
			\end{block}	

			\begin{block}{git push -u origin master}
				Przesłanie zmian na główną gałąź projektu. Można stosować samo "git push" jeżeli mamy pewność, że jesteśmy jedynymi użytkownikami repozytorium, ogólnie jest to jednak niezalecane.
			\end{block}	
			
					
						
	\end{frame}


	%------------------------------------------------

	\begin{frame}
		\frametitle{Metody pracy zespołowej}
		\begin{itemize}
			\item Dodanie "collaboratorów".
			\item Forkowanie + Pull Request.
		\end{itemize}
	\end{frame}


	%------------------------------------------------
		
	\begin{frame}
		\frametitle{Praca z kolaborantami}

			\begin{block}{git branch "Nazwa brancha"}
				Utworzenie brancha.
			\end{block}

			\begin{block}{git branch}
				Lista branchy.
			\end{block}	

			\begin{block}{git checkout "Nazwa brancha"}
				Przeskakiwanie z jednego brancha na drugi.
			\end{block}	
			
					
						
	\end{frame}
	
		%------------------------------------------------
		
	\begin{frame}
		\frametitle{Praca z kolaborantami}

			\begin{block}{git fetch "Nazwa repozytorium"}
				Pobranie nowej zawartości.
			\end{block}

			\begin{block}{git merge "Nazwa brancha"}
				Scalenie brancha do aktualnie używanej gałęzi.
			\end{block}	
			
			\begin{block}{git pull "Nazwa repozytorium"}
				Pobranie i scalenie nowej zawartości.
			\end{block}	
					
						
	\end{frame}
	
		%------------------------------------------------
		
	\begin{frame}
		\frametitle{Praca z kolaborantami}

			\begin{block}{git remote -v}
				Pokaż zdalne repozytoria.
			\end{block}	

			\begin{block}{git checkout --track "Nazwa zdalnego repo"/"Nazwa gałęzi"}
				Stworzenie śledzącej gałęzi.
			\end{block}	
			
			\begin{block}{git push "Nazwa zdalnego repo" :"Nazwa zdalnej gałęzi"}
				Usunięcie zdalnej gałęzi.
			\end{block}	
						
	\end{frame}
	
		%------------------------------------------------
		
	\begin{frame}
		\frametitle{Praca z kolaborantami}

			\begin{block}{git fetch ? git pull}
				Ostrożnie z pull'em.
			\end{block}
					
						
	\end{frame}
	
		%------------------------------------------------
		
	\begin{frame}
		\frametitle{WTH is pull request?}

			\begin{block}{Fork i pull request}
				Współpraca przez przeglądarkę.
			\end{block}								
						
	\end{frame}
	
		%------------------------------------------------
		
	\begin{frame}
		\frametitle{Ciekawe kontrolki}

			\begin{block}{git commit -a}
				Zmiany, wszędzie zmiany.
			\end{block}

			\begin{block}{git checkout -b "Nazwa brancha"}
				Skróty, skróty.
			\end{block}	

			\begin{block}{git branch --no-merged}
				Co zostało do scalenia?
			\end{block}	
								
						
	\end{frame}



	%------------------------------------------------
	
		\begin{frame}
		\frametitle{Ostatnie, ale też ważne - czyli README}
		
			Format .md jest formatem języka Markdown - języka znaczników, który służy do formatowania tekstów.\\
			Github potrafi odczytywać pliki w markdown.\\
			Rzecz jasna by Github odczytał nasze README musimy pamiętać o rozszerzeniu. Nie, żebym miała jakieś doświadczenia z tym prostym problemem...
						
	\end{frame}
	
		%------------------------------------------------
		
	\begin{frame}
		\frametitle{Podstawowe znaczniki}

			\begin{block}{\#}
				Steruje rozmiarem napisu w danej linii. Im więcej \# tym tekst jest mniejszy (mniejszy piorytet).
			\end{block}	

			\begin{block}{```}
				Służy do wyznaczania bloku tekstu. Tym znacznikiem otwieramy i zamykamy blok.
			\end{block}	
			
			\begin{block}{![screenshot.png](adres\_obrazka\_w\_naszym\_repo "Jego nazwa")}
				Wstawienie obrazka.
			\end{block}	
						
	\end{frame}
	
		%------------------------------------------------
	
	\begin{frame}
	\frametitle{Do poczytania}
	\footnotesize{
	\begin{thebibliography}{99} % Beamer does not support BibTeX so references must be inserted manually as below
	\bibitem{p1} https://help.github.com/
	\bibitem{p1} http://git-scm.com/documentation
	\bibitem{p1} http://wbzyl.inf.ug.edu.pl/sp/git

	\end{thebibliography}
	}
	
	\begin{block}{\\~\\~\\Kontakt z nami:}
	
	Karolina Rybarczyk: szprota@rybki.com.pl \\
	Radosław Głombiowski: predoobcy@gmail.com

    \end{block}
	
	
	\end{frame}

	%------------------------------------------------

	\begin{frame}
	\Huge{\centerline{Koniec}}
			
		\begin{figure}
			\centering
			\includegraphics[height=8cm]{Octocat.png}
		\end{figure}
	
	\end{frame}

	%----------------------------------------------------------------------------------------

\end{document} 
