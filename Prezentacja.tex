\documentclass{beamer}


\mode<presentation> {
	
	\usetheme[alternativetitlepage=true
		      ]{Torino}

	%\setbeamertemplate{footline} % To remove the footer line in all slides uncomment this line
	%\setbeamertemplate{footline}[page number] % To replace the footer line in all slides with a simple slide count uncomment this line

	%\setbeamertemplate{navigation symbols}{} % To remove the navigation symbols from the bottom of all slides uncomment this line
}

\usepackage[]{polski}
\usepackage[utf8]{inputenc} 
\usepackage{graphicx} % Allows including images
\usepackage{booktabs} % Allows the use of \toprule, \midrule and \bottomrule in tables

%----------------------------------------------------------------------------------------
%	TITLE PAGE
%----------------------------------------------------------------------------------------

\title[To w tym szablonie i tak nie działa]{Github, czyli tam i z powrotem z repozytorium} % The short title appears at the bottom of every slide, the full title is only on the title page

\author{Radosław Głombiowski\\oraz\\Karolina Rybarczyk} % Your name
\institute{
Uniwesytet Gdański\\ % Your institution for the title page
\medskip
\textit{john@smith.com, szprota@rybki.com.pl} % Your email address
}
\date{} % To tutaj to po to, by usunąć tą cholerną datę, która z automatu się pojawia.

\begin{document}

	\begin{frame}
		\titlepage % Print the title page as the first slide
	\end{frame}


	\begin{frame}
		\frametitle{Overview} % Table of contents slide, comment this block out to remove it
		\tableofcontents % Throughout your presentation, if you choose to use \section{} and \subsection{} commands, these will automatically be printed on this slide as an overview of your presentation
	\end{frame}

	%----------------------------------------------------------------------------------------
	%	PRESENTATION SLIDES
	%----------------------------------------------------------------------------------------

	%------------------------------------------------
	\section{Radzio - przyznam szczerze, że nie wiem czy ten slajd będzie nam potrzebny, ale na razie zostawię ten fragment kodu, tak na wszelki wypadek. Najwyżej się go wyrąbie stąd na koniec. ;D} % Sections can be created in order to organize your presentation into discrete blocks, all sections and subsections are automatically printed in the table of contents as an overview of the talk
	%------------------------------------------------


	\subsection{Subsection Example} % A subsection can be created just before a set of slides with a common theme to further break down your presentation into chunks


	%------------------------------------------------


	\begin{frame}
		\frametitle{Czym jest Github?}
		Seriws Github powstał w kwietniu 2008 roku. Jest portalem umożliwiającym hosting projektów informatycznych. Udostępnia tworzenie darmowych, publicznych repozytoriów (ma też odpłatną funkcję repozytoriów prywatnych), ułatwia pracę zespołową, prezentuje statystyki danego repozytorium oraz wiele więcej.\\
		W roku 2009 Github ogłosił, iż na stronie istnieje 90 000 unikalnych repozytoriów. Najnowsze dane (23 grudnia 2013) mówią już o istnieniu 10 000 000 repozytoriów.
	\end{frame}



	\begin{frame}
		\frametitle{Raczkowanie z Githubem}
		\begin{itemize}
			\item Założenie konta na portalu Github.
			\item Przypisanie klucza ssh do naszego konta Github.
			\item Utworzenie repozytorium na portalu oraz przyłączenie go do naszego lokalnego repozytorium.
		\end{itemize}
	\end{frame}

	%------------------------------------------------

	\begin{frame}
		\frametitle{Spis podstawowych komend do obsługi repozytorium}
			\begin{block}{git init}
				Zainicializowanie repozytorium git w folderze, w którym się znajdujemy.
			\end{block}

			\begin{block}{git add .}
				Dodaje wszystkie zmienione pliki do commita.
			\end{block}
			
			\begin{block}{git rm "nazwa pliku"}
				Usuwa podany plik z repozytorium/commita. \\UWAGA! Nie usuwa pliku z folderu a jedynie z listy plików, które będą przesłane na zdalne repozytorium.
			\end{block}			


	\end{frame}

	%------------------------------------------------
	
		
	\begin{frame}
		\frametitle{Spis podstawowych komend do obsługi repozytorium}

			\begin{block}{git commit -m "Treść"}
				Utworzenie commita. Rzecz niezbędna, w treści warto ogólnikowo podać, jakie zaszły zmiany w naszym repozytorium.
			\end{block}

			\begin{block}{git remote add origin git@github.com:"Nasz login Github"/"Nasze repo".git}
				Ustalenie konkretnego adresu dla naszego zdalnego repozytorium.
			\end{block}	

			\begin{block}{git push -u origin master}
				Przesłanie zmian na główną gałąź projektu. Można stosować samo "git push" jeżeli mamy pewność, że jesteśmy jedynymi użytkownikami repozytorium, ogólnie jest to jednak niezalecane.
			\end{block}	
			
					
						
	\end{frame}

	%------------------------------------------------

	\begin{frame}
	\frametitle{Multiple Columns}
	\begin{columns}[c] % The "c" option specifies centered vertical alignment while the "t" option is used for top vertical alignment

	\column{.45\textwidth} % Left column and width
	\textbf{Heading}
	\begin{enumerate}
	\item Statement
	\item Explanation
	\item Example
	\end{enumerate}

	\column{.5\textwidth} % Right column and width
	Lorem ipsum dolor sit amet, consectetur adipiscing elit. Integer lectus nisl, ultricies in feugiat rutrum, porttitor sit amet augue. Aliquam ut tortor mauris. Sed volutpat ante purus, quis accumsan dolor.

	\end{columns}
	\end{frame}

	%------------------------------------------------
	\section{Second Section}
	%------------------------------------------------

	\begin{frame}
	\frametitle{Table}
	\begin{table}
	\begin{tabular}{l l l}
	\toprule
	\textbf{Treatments} & \textbf{Response 1} & \textbf{Response 2}\\
	\midrule
	Treatment 1 & 0.0003262 & 0.562 \\
	Treatment 2 & 0.0015681 & 0.910 \\
	Treatment 3 & 0.0009271 & 0.296 \\
	\bottomrule
	\end{tabular}
	\caption{Table caption}
	\end{table}
	\end{frame}

	%------------------------------------------------

	\begin{frame}
	\frametitle{Theorem}
	\begin{theorem}[Mass--energy equivalence]
	$E = mc^2$
	\end{theorem}
	\end{frame}

	%------------------------------------------------

	\begin{frame}[fragile] % Need to use the fragile option when verbatim is used in the slide
	\frametitle{Verbatim}
	\begin{example}[Theorem Slide Code]
	\begin{verbatim}
	\begin{frame}
	\frametitle{Theorem}
	\begin{theorem}[Mass--energy equivalence]
	$E = mc^2$
	\end{theorem}
	\end{frame}\end{verbatim}
	\end{example}
	\end{frame}

	%------------------------------------------------

	\begin{frame}
	\frametitle{Figure}
	Uncomment the code on this slide to include your own image from the same directory as the template .TeX file.
	%\begin{figure}
	%\includegraphics[width=0.8\linewidth]{test}
	%\end{figure}
	\end{frame}

	%------------------------------------------------

	\begin{frame}[fragile] % Need to use the fragile option when verbatim is used in the slide
	\frametitle{Citation}
	An example of the \verb|\cite| command to cite within the presentation:\\~

	This statement requires citation \cite{p1}.
	\end{frame}

	%------------------------------------------------

	\begin{frame}
	\frametitle{References}
	\footnotesize{
	\begin{thebibliography}{99} % Beamer does not support BibTeX so references must be inserted manually as below
	\bibitem[Smith, 2012]{p1} John Smith (2012)
	\newblock Title of the publication
	\newblock \emph{Journal Name} 12(3), 45 -- 678.
	\end{thebibliography}
	}
	\end{frame}

	%------------------------------------------------

	\begin{frame}
	\Huge{\centerline{The End}}
	\end{frame}

	%----------------------------------------------------------------------------------------

\end{document} 